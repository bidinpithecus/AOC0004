\documentclass[12pt]{article}

\usepackage[brazil]{babel}
\usepackage[utf8]{inputenc}
\usepackage{minted}
% chktex-file 29
\sloppy

\title{Projeto PIC10F202}
\author{\sc{Pedro Faria Fernandes, Vinícios Bidin \& Yuri Becker}}
\date{01 de junho de 2023}

\begin{document}
	\maketitle

	\section{Questão 1}
		\paragraph{} Este algoritmo consiste em um somatório de 9 até 0, pois tem-se um laço de repetição com o seguinte critério de parada: o valor do registrador \texttt{mem[0x08]}
		após o decremento de 1 do valor deste, na \textit{label} `loop' precisa ser 0. Neste caso, o programa irá pular para a \textit{label} `break', onde irá pular para o final do programa.
		\par{} O literal carregado para o registrador `w', na primeira linha da \textit{label} `main' é 10, porém, como no critério de parada, antes de ser armazenado o valor da soma no registrador `w', é decrementado em um o valor de `w'. Por conta disto, se trata de um somatório do valor de $w - 1$ até 0. $\sum\limits_{i=w-1}^{1}i$
		\par{} À seguir, tem-se o equivalente ao código feito na linguagem C.
		\begin{minted}{c}
int main(void) {
	int w, f;
	f = w;
	w = 0;

	while (--f) {
		w += f;
	}

	f = w;
	return 0;
}
		\end{minted}
	\section{Questão 2}
		\begin{verbatim}
			main:
			    MOVLW 4
			    MOVWF 0x08
			    MOVLW 7
			    MOVWF 0x0C
			loop:
			    MOVF 0x08, 0
			    ADDWF 0x0C, 0
			    MOVWF 0x10

			    MOVF 0x08, 0
			    ANDWF 0x0C, 0
			    MOVWF 0x14

			    MOVF 0x10, 0
			    SUBWF 0x14, 0
			    BTFSC STATUS, 2
			    GOTO end

			    DECFSZ 0x08, 1
			    GOTO loop
			end:
			    END
		\end{verbatim}
\end{document}
